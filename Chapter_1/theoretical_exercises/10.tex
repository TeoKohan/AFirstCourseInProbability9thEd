\item From a group of $n$ people, suppose that we want to choose a committee of $k$, $k \le n$, one of whom is to be designated as chairperson.
\begin{enumerate}
    \item By focusing first on the choice of the committee and then on the choice of the chair, argue that there are $\displaystyle \binom{n}{k} * k$ possible choices.

    Hay $\displaystyle \binom{n}{k}$ formas de elegir un comité de $k$ miembros y luego $k$ formas de elegir un presidente entre los $k$ miembros.

    \item By focusing first on the choice of the non chair committee members and then on the choice of the chair, argue that there are $\displaystyle \binom{n}{k - 1} \big( n-k+1 \big)$ possible choices.

    Hay $\displaystyle \binom{n}{k - 1}$ formas de elegir un comité de $k-1$ miembros, lo que nos deja con $n-k+1$ personas entre las cuales elegir al presidente.
    
    \item By focusing first on the choice of the chair and then on the choice of the other committee members, argue that there are $\displaystyle n * \binom{n-1}{k-1}$ possible choices.

    Hay $n$ formas de elegir al presidente entre las $n$ personas y luego $\binom{n}{k-1}$ formas de seleccionar un comité de $k-1$ miembros de entre las $n-1$ personas restantes.

    \item Conclude from parts i., ii., and iii. that
    \[
        \binom{n}{k} * k = \binom{n}{k - 1} \big( n-k+1 \big) = n * \binom{n-1}{k-1}
    \]

    Se concluye inmediatamente de i., ii. and iii.

    \item Use the factorial definition of $\displaystyle \binom{m}{r}$ to verify the identity in part iv.

    \begin{align*}
        &\phantom{{}={}} \binom{n}{k} * k\\
        &= \frac{n!}{k! * (n-k)!} * k\\
        &= \frac{n!}{(k-1)! * (n-k+1)!} * \frac{n-k+1}{k} * k
        \shortintertext{lo cual concluye la primera igualdad}
        &= \binom{n}{k-1} * (n-k+1)\\
        &= \frac{n!}{(k-1)! * (n-k+1)!} * (n-k+1)\\
        &= n * \frac{(n-1)!}{(k-1)! * (n-k)!}  * \frac{n-k+1}{n-k+1}
        \shortintertext{lo cual concluye la segunda igualdad}
        &= n * \binom{n-1}{k-1} \qedhere
    \end{align*}
    
\end{enumerate}