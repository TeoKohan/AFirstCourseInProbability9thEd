\item  Consider a tournament of $n$ contestants in which the outcome is an ordering of these contestants, with ties allowed. That is, the outcome partitions the players into groups, with the first group consisting of the players who tied for first place, the next group being those who tied for the next-best position, and so on. Let $N(n)$ denote the number of different possible outcomes. For instance, $N(2) = 3$, since, in a tournament with 2 contestants, player 1 could be uniquely first, player 2 could be uniquely first, or they could tie for first.
\begin{enumerate}
    \item List all the possible outcomes when $n = 3$.

    \[
    \begin{tabular}{cccccc}
        1 1 1 & & & & & \\
        1 1 2 & 1 2 1 & 2 1 1 & & & \\
        1 2 2 & 2 1 2 & 2 2 1 & & & \\
        1 2 3 & 1 3 2 & 2 1 3 & & & \\
        2 3 1 & 3 1 2 & 3 2 1 & & &
    \end{tabular}
    \]
    \item With $N(0)$ defined to equal 1, argue, without any computations, that
    \[ N(n) = \sum_{i=1}^n \binom{n}{i} * N(n-i) \]
    \emph{Hint:} How many outcomes are there in which $i$ players tie for last place?
    
    Considerar que en todo resultado de entre las $n$ personas hay $i$ personas, $i \le n$ en la última posición. Construir todas las ocurrencias del caso donde hay $i$ personas en la última posición tomando $N(n-i)$ y seleccionando $i$ lugares entre los $n$ para ubicar a los jugadores que terminan en la última posición.
    
    \item Show that the formula of part ii. is equivalent to the following
    \[ N(n) = \sum_{i=0}^{n-1} \binom{n}{i} N(i) \]

    Inmediatamente por el mismo procedimiento construimos todas las ocurrencias del caso donde hay $n-i$ personas en la última posición tomando $N(i)$ y seleccionando $i$ lugares para que no ocupen los jugadores que terminan en la última posición.
    
    \item Use the recursion to find $N(3)$ and $N(4)$.

    Tenemos que $N(0) = 1$, $N(1) = 1$,
    \begin{align*}
        N(2) &= 3\\
        N(3)
        &= \sum_{i=1}^3 \binom{3}{i} * N(3-i)\\
        &= 3 * N(2) + 3 * N(1) + 1 * N(0)\\
        &= 9 + 3 + 1\\
        &= 13\\
        N(4)
        &= \sum_{i=1}^4 \binom{4}{i} * N(4-i)\\
        &= 4 * N(3) + 6 * N(2) + {}\\
        &\phantom{{}={}} 4 * N(1) + 1 * N(0)\\
        &= 52 + 18 + 4 + 1\\
        &= 75
    \end{align*}
\end{enumerate}